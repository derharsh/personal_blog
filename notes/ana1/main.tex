\documentclass[12pt]{article} % use larger type; default would be 10pt

\usepackage[utf8]{inputenc}

\usepackage{amsmath, amssymb, amsthm, mathtools}
\usepackage{microtype}

%%% Examples of Article customizations
% These packages are optional, depending whether you want the features they provide.
% See the LaTeX Companion or other references for full information.

%%% PAGE DIMENSIONS
\usepackage{geometry} % to change the page dimensions
\geometry{a4paper} % or a5paper or....
\geometry{margin=2.5cm} % for example, change the margins to 2 inches all round
%\geometry{landscpe} % set up the page for landscape
%   read geometry.pdf for detailed page layout information

\usepackage{graphicx} % support the \includegraphics command and options

%\usepackage[parfill]{parskip} % Activate to begin paragraphs with an empty line rather than an indent

%%% PACKAGES
\usepackage{booktabs} % for much better looking tables
\usepackage{array} % for better arrays (eg matrices) in maths
\usepackage{paralist} % very flexible & customisable lists (eg. enumerate/itemize, etc.)
\usepackage{verbatim} % adds environment for commenting out blocks of text & for better verbatim
\usepackage{subfig} % make it possible to include more than one captioned figure/table in a single float
\usepackage[hidelinks]{hyperref}
\usepackage{csquotes}
% These packages are all incorporated in the memoir class to one degree or another...

%%% HEADERS & FOOTERS
\usepackage{fancyhdr} % This should be set AFTER setting up the page geometry
\pagestyle{fancy} % options: empty , plain , fancy
\renewcommand{\headrulewidth}{0pt} % customise the layout...
\lhead{}\chead{}\rhead{}
\lfoot{}\cfoot{\thepage}\rfoot{}

%%% SECTION TITLE APPEARANCE
\usepackage{sectsty}
%\allsectionsfont{\sffamily\mdseries\upshape} % (See the fntguide.pdf for font help)
% (This matches ConTeXt defaults)

%%% ToC (table of contents) APPEARANCE
%\usepackage[nottoc,notlof,notlot]{tocbibind} % Put the bibliography in the ToC
%\usepackage[titles,subfigure]{tocloft} % Alter the style of the Table of Contents
%\renewcommand{\cftsecfont}{\rmfamily\mdseries\upshape}
%\renewcommand{\cftsecpagefont}{\rmfamily\mdseries\upshape} % No bold!

% Theorem-like environments setup
\theoremstyle{plain}
\newtheorem{axiom}{Axiom}[section]
\newtheorem{theorem}{Theorem}[section]
\newtheorem{lemma}[theorem]{Lemma}
\newtheorem{proposition}[theorem]{Proposition}
\newtheorem{corollary}[theorem]{Corollary}

\theoremstyle{definition}
\newtheorem{definition}[theorem]{Definition}
\newtheorem{example}[theorem]{Example}
\newtheorem{remark}[theorem]{Remark}
\newtheorem{notation}[theorem]{Notation}

% Custom theorem style for important results
\newtheoremstyle{important}
  {\topsep}   % Space above
  {\topsep}   % Space below
  {\itshape}  % Body font
  {}          % Indent amount
  {\bfseries} % Theorem head font
  {.}         % Punctuation after theorem head
  {.5em}      % Space after theorem head
  {\thmname{#1}\thmnumber{ #2}\thmnote{ (#3)}} % Theorem head spec

\theoremstyle{important}
\newtheorem*{maintheorem}{Main Theorem}

% Remove extra indentation after environments
\AtEndEnvironment{axiom}{\noindent}
\AtEndEnvironment{theorem}{\noindent}
\AtEndEnvironment{lemma}{\noindent}
\AtEndEnvironment{proposition}{\noindent}
\AtEndEnvironment{corollary}{\noindent}
\AtEndEnvironment{definition}{\noindent}
\AtEndEnvironment{example}{\noindent}
\AtEndEnvironment{remark}{\noindent}
\AtEndEnvironment{notation}{\noindent}
\AtEndEnvironment{maintheorem}{\noindent}

%%% END Article customizations

\title{\textbf{Analysis I}}
\author{Harsh Prajapati}
\date{23.03.26}

\begin{document}
\maketitle

\noindent These notes were prepared between October 2025 and March 2026 \textbf{(Last update: \today)}.

\noindent If you find any mistakes or typos, please report them to \textbf{caccacpenguin@gmail.com}. I would really appreciate it.

\noindent I often use informal language to make the ideas easier to grasp. My goal is to make the material feel approachable, while still respecting the rigor that makes mathematics what it is.

\noindent I hope you will find these notes helpful :D!

\subsection*{References}

These lecture notes closely follow these two lecture notes'. However, they are written in German.

\begin{itemize}
    \item Analysis I --- WiSe 2016/17, by Franz Merkl, Facult\"at f\"ur Mathematik, Informatik und Statistik, LMU M\"unchen.
    \item Analysis einer Ver\"anderlichen ---  WiSe 2013/14, by Lars Diening, Facult\"at f\"ur Mathematik, Informatik und Statistik, LMU M\"unchen.
\end{itemize}

\noindent These books will serve as great references. There's a mix of book English and German texts, although you can find English translations for some of them.

\begin{itemize}
    \item Herbert Amann, Joachim Escher, Analysis I, Dritte Auflage (My Table of Content follo)
    \item Otto Forster, Florian Lindemann, Analysis 1, 13. Auflage
    \item Walter Rudin, Principles of Mathematical Analysis, 3rd. Edition
    \item K. Königsberger, Analysis 1
    \item W. Walter, Analysis 1
    \item Serge Lang, Analysis I
\end{itemize}

\subsection*{My Other Notes}

\textbf{Mathematics:}

\begin{itemize}
  \item Foundation of Mathematics - \textit{Logic, Set Theory and Proofs}
  \item \textbf{Analysis I} - \textit{Analysis of Functions of Single Variable}
  \item Analysis II - \textit{Differential Calculus of Several Variables}
  \item Analysis III - \textit{Measure Theory, Integral Calculus, and Vector Analysis}
  \item Linear Algebra I - \textit{Algebraic Foundations, Vector Spaces, Matrix and Eigenvalue Theory}
  \item Linear Algebra II - \textit{Canonical Forms, Inner Product Space, Bilinear forms}
  \item Stochastic - \textit{Probabilty, Statistics and Applications}
  \item Algebra - \textit{Groups, Rings, Fields and Modules}
  \item Ordinary Differential Equations
  \item Function Theory - \textit{Analysis of Functions of Complex Variables}
  \item Functional Analysis I - \textit{General Theory of Functional Spaces}
  \item Functional Analysis II - \textit{Advanced Theory of Functional Spaces}
  \item Geometry - \textit{Foundations of Geometry and Topology}
  \item Probability Theory
  \item Differential Geometry
  \item Sympletic Geometry
  \item Riemannian Geometry
  \item Partial Differential Equations I
  \item Partial Differential Equations II
  \item Commutative Algebra
  \item Algebraic Number Theory - \textit{Number Fields, Valuations and Adeles}
  \item Algebraic Topology - \textit{Homotopy, Homology and Cohomology Theories}
  \item Stochastic Analysis and Differential Equations
  \item Algebraic Geometry - \textit{Varieties, Schemes and Sheaves}
\end{itemize}

\textbf{Mathematical Physics:}

\begin{itemize}
  \item Classical Mechanics I - \textit{Newton, Hamilton and Lagrangian formalism}
  \item Classical Mechanics II - \textit{Sympletic Geometry, Qualitative Dynamics, Potential Theory, and Celestial Mechanics}
  \item Classical Mechanics III - \textit{Continnum Field Theory on Infinite-Dimensional Manifolds and Euler-Poincaré Reduction}
  \item Quantum Mechanics I - \textit{Canonical Quantisation, Path Integrals, Abstract Hilbert Space and Functional Analytic Foundations}
  \item Quantum Mechanics II - \textit{Measurement and Supersymmetry}
  \item Classical Electrodynamics - \textit{$U(1)$ Gauge Theory on Fibre Bundles}
  \item Statistical Physics - \textit{Probabilitic Dynamical Systems and Ergodic Theory}
  \item Theory of Relativity - \textit{Theory of Lorentzian and Semi-Riemannian Geometry}
  \item Quantum Field Theory I - \textit{Path Integrals, EFT, Gauge Symmetry, QED, and The Standard Model}
  \item Quantum Field Theory II - \textit{Axiomatic and Constructive QFT}
  \item Quantum Field Theory III - \textit{Stochastic Quantisation and Regularity Structures in SPDEs}
  \item Mathematical Gauge Theory
\end{itemize}\clearpage

\tableofcontents
\clearpage

\section{Introduction}

You have already learned how to do calculus and somewhat understand why it works. So, you might wonder, \enquote{\emph{Why study calculus again? Why do we need to justify it formally?}}

It's a fair question. We already have an intuitive idea of how \emph{infinitesimal calculus} works: the derivative is defined as the slope of a secant line on a curve when the points on the curve come \enquote{infinitesimally close} to each other, so close that they almost overlap but are still two distinct points.

But have you ever wondered what \enquote{infinitesimally close} or \enquote{infinitesimally small} quantities even mean? You might say, \enquote{It means that the quantity is \emph{very small}, almost close to zero but not zero \emph{itself}, such as $10^{-40}$}.

But examples like that describe \emph{concrete} small quantities, it is a \emph{concrete} value of \enquote{how small}. No matter how small a quantity you pick, \enquote{infinitesimally small} quantities are supposed to be always smaller than them.

So, infinitesimals seems like this vague, intuitive idea which we can't pin down precisely. You may wonder, \enquote{How do you \emph{precisely} pin down something?}

See, in mathematics, we try to prove statements from first principles formally, and visual demonstrations or informal intuitive explanations are not accepted as proofs.

This topic was discussed in more detail from a philosophical perspective in the notes on \textit{Foundations of Mathematics} and the crux of it is this: our informal intuitions are based on assumptions that are not explicitly pinned down, in formal mathematics we try to pin down every assumption explicitly.

So, the reason for studying analysis is not just to understand why calculus works and when our rules apply, it is to understand what \emph{formally proving} something even means and why we do it.

\enquote{\emph{Okay, so you're saying that our informal intuitions hide some assumptions but what exactly are those assumptions?}}

The assumptions for calculus actually start from very basic concepts in maths, from numbers and arithmetic itself. Because we deal with real numbers in analysis we need to discuss what real numbers mean and construct it axiomatically from primitive axioms. The need for this is that there is some ambiguity in our understanding of numbers, for example, do you know if 0.999\dots \emph{exactly} equals 1 or is this just an approximation? And does infinite sum of a geometric series exactly equal to its limit? We say that it \enquote{approaches this value}, but why souldn't it just keep on increasing forever, even if by extremely tiny amounts?

Our aim is to resolve such doubts and gaps in our understanding caused by the intuitive and imprecise explanation given in school. For that purpose, we are usually given the $\delta-\epsilon$ arguments for a limit and convergence, and we use similar arguments for derivatives and integrals as well.

But, if we should accept that with all these $\delta-\epsilon$ arguments convergence happens, then why are we introduced to the notion of topology later on, if convergence was already justified formally? You're usually told that, \enquote{These ideas were way too abstract for that time, so we just stuck to proving $\delta-\epsilon$.} But that's not true at all, if you understand the need for topology, the idea of neighbourhoods, closure, compactness will not seem like arbitrary abstract ideas but the inevitable need to justify Newton and Leibniz.

The reason why analysis is so hard even for the best students is not just because it's the first time dealing with abstraction and proofs, but also that students are not given proper motivation. I will not try to rush through the concepts or delay topological foundations just for the sake of pedagogy but without honesty. I'll show you that analysis is not \enquote{Calculus with Proofs} but \enquote{Justification of Calculus}, a subtle but important distinction.

The prerequisites for this course are working knowledge of logic, sets and proof techniques, you can refer to my notes on \textit{Foundations of Mathematics} for those topics. Those notes cover much more than we need for this course but you can skip some sections and cover the essentials.

Some knowledge of groups, rings, fields and vector spaces would also be helpful but that can be covered alongside this course. Apart from these, only high school mathematics is expected from the reader.

In the first volume of the notes on Analysis we'll focus on the analysis of real numbers and functions of single real variables. Although we will discuss some basic concept of functions of complex variables and how functions in general form vector spaces as well, a detailed coverage will not be done here. You can refer to the notes on \emph{Function Theory} and \emph{Functional Analysis} for a detailed coverage. For applications to theoretical physics, calculus of vector-valued functions and a section on Fourier Analysis is also given.

%In the second volume we'll generalise the concepts to multivariable functions and discuss about manifolds which is the general object over which we do calculus. We'll also discuss how the metric tensor helps normalisation of basis vectors under coordinate transformation for applications to physics.

%In the third volume, we'll see how some functions are not integrable using the Riemann integral and develop a more general perspective of integration as a measure for measure theory and advanced analysis.

\clearpage
\section{Numbers and Fields}

To discuss analysis rigorously, we need to start from numbers. You might feel a bit impatient, asking, \emph{\enquote{Why should we talk about numbers first? Don't we already understand numbers? And what does it have to do with calculus?}}

You are right to question this. Mathematicians at a time also believed that they understood everything about numbers, they knew the rules to manipulate numbers and do operations on them. But this belief crumbled in the 19th century.

In this chapter we will rigorously construct the number systems from first principles in the hierarchy of the number sets $ {\mathbb{N}}\subset{\mathbb{Z}}\subset{\mathbb{Q}}\subset{\mathbb{R}}\subset{\mathbb{C}}$, i.e. the \emph{natural}, \emph{integers}, \emph{rational}, \emph{real}, and then the \emph{complex} numbers. Each successive number set can be constructed from the previous one, and we'll see how.

We'll see both the \emph{constructive} and \emph{axiomatic} approaches, but our primary focus would be on the constructive approcah which has the advantage over the \emph{axiomatic} formulation of the real numbers of David Hilbert, that the entire structure of mathematics can be built up from a few foundation stones coming from mathematical logic and axiomatic set theory.

\subsection{Structure of Numbers Systems}

Even though in analysis courses we construct the number systems from natural numbers to complex in the hierarchy discussed above, this is not the way they were historically developed. Apart from the natural numbers other numbers such as fractions and irrational numbers such as $\pi$ and square roots were also being used long before, even in ancient civillisations.

To really understand how numbers emerged and evolved to the modern system, let's go all the way back. You might wonder, \enquote{\textit{Is this a history class? Why can't we just start with the maths?}}

Don't worry, I won't spend too much time on the history, the goal is to see how number systems evolved and what was the philosophy behind it. Although you can skip this section if you want.

\subsubsection{Historical Development}

Natural numbers emerged \enquote{naturally} due to the need of ancient humans to count \emph{real discrete} objects in the world such as apples, trees and animals. You can find use of notches on bones and marks on the walls of caves as \enquote{symbols} for numbers even in the \emph{early stone age}.

But the first systemic use of numbers in civillisations were found in the \emph{valley of the Nile, Euphrates and Tigris}. \emph{Egyptians} for instance made use of \textbf{hieroglyphs}\index{hieroglyphs} for powers of 10 such as the numbers 1, 10, 100, 1000, 10 000, 100 000 and 1 000 000 \footnote{found on a mace of \textbf{King Narmer}, of the first Egyptian dynasty (circa 3000 BC)}.

The \emph{Babylonians} used \textbf{cuneiform symbols} on clay tablets, based on a mixed decimal and sexagesimal position notation which was more complicated than the Egyptians. These symbols looks completely different to the ones we use today but they were enough to form any natural number (except zero) and they could also do addition, subtraction, multiplication and divisions with them, and not only that, they even had rules to form fractions, do operations and solve equations with them.

Babylonians in particular were quite skillful and had sophisticated techniques for arithmetic and algebra, they had considerable influence on the development of these areas.\footnote{for a detailed account see H.-D. Ebbinghaus, H. Hermes, F. Hirzebruch, M. Koecher, K. Mainzer, J. Neukirch, A. Prestel, R. Remmert, \emph{Zahlen}}.

The \emph{earlier system} of the \emph{Greeks} made use of symbols in a decadic number system which could be used to represent numbers such as 1, 5, 10, 50, 100, 500, and other multiples of them.

You have probably recognised the pattern here, every civillisation is creating a system which assigns some set of symbols to \emph{some} numbers which could be used to build \emph{other} numbers, but so far we haven't seen unique symbols for every number between 1-9 like the modern system.

The \emph{later Greek system} made use of the 24 letters of the standard Greek alphabet and three more from the oriental tradition to represent every number between 1-9, 10-90 (multiples of ten: 10, 20,\dots), 100-900, 1000-9000\footnote{strictly speaking the same symbols for 1-9 were used with a subscript accent on the left for 1000-9000.} and 10,000. This is sometimes also called the Ionic system. It was probably the first time when writing numbers wasn't so tedious but calculations in the Greek system was still quite complicated since it wasn't purely positional.

\subsubsection{Philosophy of Numbers}

Even though I used the modern numbers (1, 2, 3, etc.) as a means to explain what these systems represented, you should not think of those symbols as representing \enquote{one}, \enquote{two}, \enquote{three} and so one, it's much better to think of it as a \emph{one-to-one correspondence} of representing some \enquote{amount} of discrete objects by those symbols.

This could be hard to get your head around since we are so used to connecting the idea of the words \enquote{one}, \enquote{two}, \enquote{three} to the idea of counting. So, let me give you a famous example: A flock of four sheep and a grove of four trees are related to each other in a way in which neither is related to a pile of three stones or a grove of seven trees. Of course, you know the relationship here is about the 'number' of objects but we don't need to invoke numbers to actually understand the relationship. The relationship which is being refered to is the concept of \textbf{cardinal numbers}\index{cardinal numbers}. Without counting the sheep or the trees, we can pair them with each other, for example by tethering each (and only one) sheep to the trees, so that each sheep and each tree belongs to exactly one of the pairs. Such a pairing between the members of two sets of objects is called a \textbf{one-to-one (1-1) correspondence}\index{one-to-one correspondence}.

This is why even animals and toodlers can tell the difference between the amount of discrete objects, numaricity is a pre-linguistic concept and we don't need human language for it \footnote{see \emph{Approximate Number System} for a detailed account of the study on how animals and toddlers also have an idea of cardinal numbers}.

You can find a representation of numbers by counters by the Greeks (such as the beads of an abacus, pebbles and so on), which was a means by which arithmetical theorems were discovered.

So, the first step is to separate the idea of Numerosity, the concept of “how many” in quantity, from the concept of Numeral, the symbolic representation.

\subsubsection{Formalism of Numbers}

The technical term for this concept of 1-1 correspondence is \textbf{cardinality}, it was used by Frege and Cantor to define natural numbers as \enquote{finite potencies} and \enquote{finite cardinal numbers} respectively. A similar concept was used by B. Russell and N. Bourbaki as well. We'll see how the cardinality of sets can be used to formulise numbers in later sections.

If you are thinking that, \enquote{\emph{OK, so numbers are just 1-1 correspondence like mapping.}} Then, it's not really true. Some people argue that the concept of cardinality or any sort of correspondence is not really necessary to define numbers, it can be constructed purely from formal axioms as purely abstract idea.

This purely abstract idea of numbers is fine if you just want to do maths. In fact, both of these perspective have thier pros and cons depending on what you are working on, the axiomatic formulation does not care about what the natural numbers are, it just only cares about what are the properties they and what you can do with them. This might seem philosophically dull and it sort of is, but it is actually quite helpful when moving from standard to non-standard numbers and we'll discuss it in further sections how.

You may be a bit annoyed, thinkinh, \enquote{\emph{Why can't mathematicians just have one formulation? Why does everyone have different opinions on numerbs? Isn't maths supposed to be objectively true?}}

I get what you feel, it does seem like mathematicians don't seem to agree even on numbers, this is mostly because different people came up with different formulations of number systems but the good part is that all of them are actually equivalent.

It was not until the 19th century that mathematicians gave formal definitions of the concept of number, and their foremost consideration was initially to provide \emph{secure foundations for analysis}, which is why we are discussing about numbers so much. It was not until after Dedekind and Cantor (and others) had defined real numbers by means of sets of rational numbers that the classical definitions of the natural numbers in terms of logic and set theory then followed. The realization that the extensions of the natural numbers to the integers and the rationals could still essentially be regarded as a topic of algebra was closely bound up with the introduction of the fundamental algebraic ideas of ring theory and field theory

%But before going further, after defining the natural numbers we'll discuss the rules of arithematics and why the rules of algebra work. And we'll proof every rule rigorously. This will leave no room for ambiguity and we'll have rigorously defined everything.

%\emph{“Okay, but how does that help us?”}

%We are not trying to be pedantic here.

%Analysis is really a step up of Arithematics, it's not merely just “Calculus with Proofs”. The reason why we're talking about natural number so much is because we need to build all the way up to the reals to talk about functions of real variables and do calculus on them. This is why the entire first chapter is dedicated to just the natural numbers.

\subsection{Natural Numbers}

We already have an intuitive idea of what the natural naturals are: elements of the set \(\mathbb{N} := \{0,1,2,3,\dots\}\)\footnote{in some texts the natural numbers start at 1 instead of 0, but this is just a matter of convention. I believe that it philosophically makes sense to include zero to have a concrete notion of \enquote{nothing} in quantities.}. But this definition is not adequate, we don't precisely know what the set $\mathbb{N}$ \emph{is}. What I mean by that is we are constructing the set of natural numbers by using the natural numbers \emph{themselves}.

This definition of the natural numbers is just like saying \enquote{start from 0 and count forward indefinitely}. But this idea begs the question, \enquote{How do we know we can count forward \emph{indefintely} without ending up with some largest number? Or circuling back to some other number say 0 itself?} This question might seem a bit silly but we'll see that if you don't choose your axioms properly then circuling back to 0 could be possible.

\subsubsection{The Peano Axioms}

The goal is to define the natural numbers using the most primitive principles. A standard way to construct the natural numbers is the based on the axioms of Guiseppe Peano, who formalises the idea that given any natural number, there is always a next largest natural number.

Historically this is not how natural numbers were formulated the first time but we want to be able to derive the rules of arithematics just from these axioms and it is easier for beginners to understand in this way.

Unlike Dedekind, Peano was not interested in set theoretical construction of natural numbers but in axiomatisation in formal language. But, I will still make extensive use of set theory and logic just for the sake of rigour, so make sure you have the necessary background.

From the conclusion of our previous discussions, it seems like the natural number consist of the set $\mathbb{N}$, with 0 as a distinguised element and we can get every other natural number by counting forward. \enquote{Counting forward} can be defined as an increment operation or more precisely a \textbf{successor function}\index{successor function} $\nu:\mathbb{N}\rightarrow\mathbb{N}$, thus for $n\in\mathbb{N}$, the element $\nu(n)$ is called the \textbf{successor}\index{successor} of $n$. So, we have the objects: 0, $\nu(0)$, $\nu(\nu(0))$, $\nu(\nu(\nu(0)))$,\dots as the elements of $\mathbb{N}$. Of course, writing the elements of $\mathbb{N}$ in this way can get very unweirdy, hence we'll adapt a different convention, and write in the familiar notation: $1:=\nu(0), 2:=\nu(1), 3:=\nu(2),\dots$

If we start writing the axioms for natural numbers it seems like we only need these two axioms to define natural numbers formally:

\begin{axiom}
Zero is a natural number. \emph{In formula}: ${0}\in{\mathbb{N}}$.
\end{axiom}

\begin{axiom}\label{ax: succ_fun}
If $n$ is a natural number, then $\nu (n)$ is the successor of $n$ and it is also a natural number. \emph{In formula:} \(n \in \mathbb{N} \implies \nu(n) \in \mathbb{N}\)
\end{axiom}

And now we prove for any property of the natural numbers, let's take up an example.

\begin{proposition}
    4 is a natural number.
    \begin{proof}
        We have from Axiom 2.1, $0\in{\mathbb{N}}$, and ${n}\mapsto{\nu(n)}$ from Axiom 2.2. To show is $4\in{\mathbb{N}}$. From Axiom 2.1 and Axiom 2.2, we get ${0}\mapsto{\nu(0)=1}$, ${1}\mapsto{\nu(1)}=2$, ${2}\mapsto{\nu(2)}=3$, and ${3}\mapsto{\nu(3)=4} \Rightarrow {4}\in{\mathbb{N}}$, which was to be shown.
    \end{proof}
\end{proposition}

It might seem like these two axioms are enough but there are some problems. In previous sections I said that if you don't choose your axioms properly than it is possible to circle back at some number. Since, I did not say whether $4:=\nu(3)$ it is fair to ask how I could have come to this conclusion and not get something like 1 or 2, since the function $\nu$ is not injective it would not contradict Axiom 2.1 and 2.2 if $\nu(3)=1$.

To avoid this issue, we simply define $\nu$ to be an injective function and propose this axiom:

\begin{axiom}
    If \(n, m \in \mathbb{N}\) are two different natural numbers, i.e. \({n}\neq{m}\) then \({\nu(n)}\neq{\nu(m)}\). \emph{In formula}: $$\forall{n, m\in\mathbb{N}}:{({\nu(n)}={\nu(m)})}\Rightarrow{({n}={m})}\footnote{this is by using the contraposition of the implication, see \emph{Foundations of Mathematics} for basics of logic.}.$$
\end{axiom}

Now, two numbers cannot have the same successor. Now, to address the other issue, by following our axioms, we could get $\nu(n)=0$ for some $n\in{\mathbb{N}}$, for example if we had $\nu(4)=0$ then we would be circuling from 0,1,2,3,4, to 0 again, which would lead to 0,1,2,3,4,0,1,2,3,\dots which is exactly what I meant by circling back to 0.

To avoid this we shall say that zero is not a successor of any number. Formally:

\begin{axiom}
    No natural number has 0 as the successor. \emph{In formula}: $$\forall{n\in{\mathbb{N}}}:{\nu(n)}\neq{0}.$$

    \emph{Equivalently}: ${0}\notin{\nu[\mathbb{N}]}$.
\end{axiom}

Another way to deal with this issue is to redefine the successor function itself as ${\nu}:{\mathbb{N}}\rightarrow{\mathbb{N}^{\times}}$, where $\mathbb{N}^{\times }:= \mathbb{N}\backslash\{0\}$. This makes $\nu$ a bijection and restricts 0 from being a successor of any other number since it's not even in the codomain of the mapping $\nu$, our explicit axiom does this thing in a much simpler way and we can prove any proposition related to natural numbers.

\begin{proposition}
    1 is not equal to 5.
    \begin{proof}
        To show is ${\lnot(1=5)}\Leftrightarrow{[{(1=5)}\Rightarrow{\bot }]}$. Let, $1=5$. It follows from Axiom 2.2 and 2.3 $${(1=5)}\Rightarrow{(\nu(0)=\nu(4))}\Rightarrow{(0=4)}.$$
        
        From Axiom 2.4, ${[{(1=5)}\Rightarrow{(0=4)}]}\Leftrightarrow{\bot}$. Hence, $1\neq5$, which was to be shown.
    \end{proof}
\end{proposition}

So, it seems like now we've fixed all problems and now our axioms describe the behaviour of natural numbers perfectly. But there is still one issue, we are allowing some Terms\footnote{see the notes on \emph{Foundations of Mathematics} for basic concepts of Type Theory} which may not be of the Type ${\mathbb{N}}$. Meaning there may be other “rogue” elements in our number system which are not of the form 0, 1, 2, 3,\dots Because I did not write out the \enquote{\dots} in Axiom~\ref{ax: succ_fun}, we don't know if the \enquote{pattern} of the symbols 0, 1, 2, 3, would \emph{for sure} continue in the fashion we want. You could have 0, 1, 2, 3, $\phi$, $a$, $\pi$ and whatever, and it would still satify all our axioms.

What we want \emph{precisely} is that the 1-1 correspondence between the \emph{objects} 0, $\nu(0)$, $\nu(\nu(0))$, $\nu(\nu(\nu(0)))$,\dots (which were in $\mathbb{N}$ due to the successor function) and the \emph{symbols} 0, 1, 2, 3,\dots (which were \emph{not} in $\mathbb{N}$, except 0) should continue, but we did not formally guarantee it for all objects in $\mathbb{N}$\footnote{you may wonder that this would mean that the deduction in the proofs might be invalid but that's not true because withing our axiom system the conclusions do not create any contradicions.}.

This could be solved by introducing type theory concepts but we want to stick to the historical formalism, we will explore this modern idea later on.

This seems like a very difficult task to do, we would have to write out all the correspondence explicitly and that is very impractical. So, we will use a simple but powerful technique:

\begin{axiom}[Induction Schema]
    For every property $\varphi(n)$ over any natural number $n$ it holds that: if $\varphi(0)$ is true, and if for all $n \in \mathbb{N}$, the property $\varphi(\nu(n))$ follows from $\varphi(n)$, then $\varphi(n)$ is true for all $n \in \mathbb{N}$.

    \emph{In formula}: 
    \[
    \bigl[\varphi(0) \wedge \forall n \in \mathbb{N}\,:(\varphi(n) \Rightarrow \varphi(\nu (n)))\bigr]
    \;\Rightarrow\;
    \forall n \in \mathbb{N}\,:\varphi(n)
    \]
\end{axiom}

\begin{remark}
    This axiom is stated as a \emph{schema} because it represents an infinite family of axioms - one for each property $\varphi$ of the natural numbers. More generally, you can say that the induction schema is a pattern of formulas that yield infinitly many axioms having the formula ${[{\varphi (0)}\land{\forall{x}}:{({\varphi (x)}\Rightarrow{\varphi (\nu (x))})}]}\Rightarrow{\forall{x}}{\varphi (x)}$ each formula $\varphi (x)$ in the language of arithematics.\footnote{This is just one of the forms of the \textbf{induction schema}\index{induction schema}.}
\end{remark}

This schema says that if some property $\varphi(n)$ holds for 0, meaning $\varphi(0)$ is and if \(\varphi(n)\) is true for every natural number, then \(\varphi(\nu(n))\) is also true for all natural numbers, and if both of them are true then \(\varphi(n)\) holds for all natural numbers. Hence, we can get:

\[
\varphi(1) \;\text{because}\; \varphi(0) \Rightarrow \varphi(\nu(0))
\]
\[
\varphi(2) \;\text{because}\; \varphi(1) \Rightarrow \varphi(\nu(1))
\]
\[
\varphi(3) \;\text{because}\; \varphi(2) \Rightarrow \varphi(\nu(2))
\]
\[\vdots \]

and after recursive application of the induction schema for infinitly many steps, we get - \(\forall n \in \mathbb{N} : \varphi(n)\). This avoids any \enquote{rogue} objects to appear.

The Axioms 2.1-2.5 are known as the \emph{Peano axioms} for the natural numbers. Historically Peano originally formulated nine axioms (with 1 as the distinguised element)\footnote{see, Guiseppe Peano, \emph{Arithmetices Principia nova methodo exposita}, 1889}, and as I already said, he did not use set theory rather used second-order logic. The first five were the same as ours and the next four were \emph{logical axioms} about equality.

\subsubsection{Definition of Natural Numbers}

The Peano axiomatisation postulates axioms that any model of the natural number must satify. So far, we have only chosen axioms based on informal reasoning, but we have not provided a formal proof that such a model actually exists.

We will now explore Dedekind's approach and other investigations into proving the existence proof.

To do this, we first define the natural numbers in a modern set-theoretic way. As mentioned before, this definition is equivalent to the Peano Axioms:

\begin{definition}
    The \textbf{natural numbers} are defined as a triple \((\mathbb{N}, 0, \nu)\), consisting of a set \(\mathbb{N}\), with a distinguished element \(0 \in \mathbb{N}\), together with a successor function \(\nu : \mathbb{N} \rightarrow \mathbb{N}\) which satisfy the following axioms:
    \begin{itemize}
        \item \(\nu\) is injective
        \item \(0 \notin \nu[\mathbb{N}]\)
        \item If \(N \subset \mathbb{N}\, , 0 \in N,\) and \(\forall n \in \mathbb{N} : \nu(n) \in N\) then \(N = \mathbb{N}\)
    \end{itemize}
\end{definition}

The third property is the set theoretic formulation of the \emph{principle of complete induction}, its equivalent to the Axiom 2.5. can be seen if you replace the property \(\varphi\) by the subset $N$. The principle of induction is not some new kind of syllogism of mathematicians set apart from the ordinary rules of inference in logic; it is merely the use of the third axiom to prove that certain statements are valid for all natural numbers.

This is an axiomatic definition of the natural numbers, meaning we haven't yet \emph{constructed} the familiar \{0,1,2,3,\dots\}, we have only stated that there exists a system \((\mathbb{N}, 0, \nu)\) in which the Peano axioms hold. You could use either the Indo-Arabic system \{0,1,2,3,\dots\} or the Roman system \{O,I,II,III,\dots\} and both will satify the Peano axioms, in fact these systems are not really different except for the use of different symbols, so we have an \textbf{isomorphism} between these two systems and certainly any system \((\mathbb{N}, 0, \nu)\) which satifies the Peano axioms will be isomorphic to any of these two, we will show this isomorphism rigorously in a moment.

First let's consider whether we can show that there exists a \textbf{model} for the natural numbers. Clearly if the natural numbers exists they must form an \emph{infinite system: A set $M$ is called an infinite system, if there is an injective mapping \(f : M \rightarrow M\) such that \(f[M] \subset M\)}.

This definition expresses the fact that only infinite sets can be mapped injectively onto one of their proper subsets. Historically this was the definition given by Dedekind, though instead of injective mappings, he used the term \enquote{ähnliche Abbildungen} (similarity mappings)\footnote{see, Richard Dedekind, \emph{Was sind und was sollen die Zahien?}}. The significance of such system is the following theorem proved by Dedekind:

\begin{theorem}
    Any infinite system contains a model \((\mathbb{N}, 0, \nu)\) for the natural numbers.\emph{In other words:} There exists an infinite set if and only if there exists a system \((\mathbb{N}, 0, \nu)\).
    \begin{proof}
        Let $A$ be an infinite system. Then by definition there is an injective mapping \(f : A \to A\) with \(f[A] \subset A\). It follows that \(0 \in A \implies f(0) \notin A\). Let $I$ be the class of all sets \(M \subset A\) with \((0 \in M) \wedge (f[M] \subset M)\). By hypothesis, \(I \neq \emptyset \). Thus we can define the intersection \(\bigcap_{M \in I} M\). This set satifies the axioms for \((\mathbb{N}, 0, \nu)\) if one takes \(f | M\) as the mapping \(\varphi\).
    \end{proof}
\end{theorem}

Thus to prove the existence of a model we just need to prove the existence of an infinite system.

Dedekind gave an existence proof which implicitly used the \textbf{unrestricted comprehension principle}\index{unrestricted comprehension principle} introduced by G. Frege in 1893: \emph{For all property $\varphi$ of sets, the set \(M _ \varphi := \{x |\, \varphi(x)\}\) exists.} But as you may know, B. Russell found that this axiom leads to contradictions. Similar unsuccessful attempt was made by Bolzano\footnote{see, Blozano, \emph{Paradoxien des Unendlichen}}.

So to prove existence of naturanl numbers in the framework of axiomatic set theory, we assume the \emph{restricted comprehension principle}, and we need the \emph{Infinity axiom: An Inductive set exists.} Here an \textbf{inductive set}\index{inductive set} is a set $N$ which contains $\emptyset$, such that for all \(z \in N\), \(z \cup \{z\}\) is also in $N$. We can thus use this inductive to form an infinite system: \[\mathbb{N} := \{m | m \;\text{is an inductive set}\},\] now we define our successor mapping \(\nu : \mathbb{N} \to \mathbb{N}\) by \(\nu(z) := z \cup \{z\}\). Now, we'll define \(0 := \emptyset\). This construction can now shown that $\mathbb{N}$ is itself an inductive set and our system  \((\mathbb{N}, 0, \nu)\) satisfies the Peano axioms. Thus \((\mathbb{N}, 0, \nu)\) is a model for the natural numbers. Below I'll give an informal explanation of how this works:

We start with \(0 := \emptyset\), then we use the successor mapping and get

\[
1 := \nu(0) = 0 \cup \{0\} = \emptyset \cup \{\emptyset\} = \{\emptyset\} \;\text{so}\; 1 := \{0\},
\]
\[
2 := \nu(1) = 1 \cup \{1\} = \{\emptyset\} \cup \{\{\emptyset\}\} = \{\emptyset, \{\emptyset\}\} \;\text{so}\; 2 := \{0,1\},
\]
\[
3 := \nu(2) = 2 \cup \{2\} = \{\emptyset, \{\emptyset\}\} \cup \{\{\emptyset, \{\emptyset\}\}\} = \{\emptyset, \{\emptyset\}, \{\emptyset, \{\emptyset\}\}\} \;\text{so}\; 3 := \{0,1,2\},
\]

And since we have already said that \(\mathbb{N}\) is an inductive set, every natural number is defined as the set of numbers smaller than it. Basically, we already assumed that infinite sets exist and in our proof, \(\mathbb{N}\) was the smallest infinite set.

\subsubsection{Arithematics of Natural Numbers}

\subsubsection{The Division Algorithm}

\subsubsection{Complete Induction}

\subsubsection{Recursive Definition}

\subsection{Integers and Rationals}

\subsubsection{Integers}

\subsubsection{Rational Numbers}

\subsubsection{Rational Zeros of Polynomials}

\subsubsection{Absolute Value, Exponentials, and Square roots}

\subsubsection{Gaps in Rational Numbers}

\subsection{The Field Axioms}

\subsection{The Real Field}

\subsubsection{Order Completeness}

\subsubsection{Dedekind Cuts}

\subsubsection{\texorpdfstring{The Natural Order on $\mathbb{R}$}{The Natural Order on R}}

\subsubsection{The Extended Number Line}

\subsubsection{A Characterization of Supremum and Infimum}

\subsubsection{The Archimedean Property}
\subsubsection{The Density of the Rational Numbers in R}
\subsubsection{nth Roots}
\subsection{The Density of the Irrational Numbers in R}

\subsubsection{Intervals}

\subsection{The Complex Field}

\subsection{\texorpdfstring{Balls in $\mathbb{K}$}{Balls in K}}

\subsection{p-Adic Numbers*}
\clearpage
\section{Foundations of Topology}

\subsection{Topological Spaces}

\subsection{Metric Space}

\subsection{Compactness}

\subsection{Connectivity}

\subsection{The Hausdorff Condition}
\clearpage
\section{Sequence}

\subsection{\texorpdfstring{Convergence of Sequence in $\mathbb{R}$ and $\mathbb{C}$}{Convergence of Sequence in R and C}}

\subsection{Cauchy Sequences}

\subsection{Infinite Limits}

\subsection{Completeness}

\clearpage
\section{Series}

\subsection{Finite and Infinite Series}

\subsection{Calculations with Series}

\subsection{Majorant, Root and Ratio Tests}

\subsection{Absolute Convergence}

\subsection{Rearrangement of Series}

\subsection{Power Series}

\clearpage
\section{Continuous Functions}
\subsection{\texorpdfstring{Limits and Continuity of Functions in $\mathbb{R}$}{Limits and Continuity of Functions in R}}
\subsection{Compactedness and Connectedness}
\subsection{Monotonic Functions}
\subsection{Limits at Infinity}
\subsection{\texorpdfstring{Exponential and Related Functions in $\mathbb{R}$}{Exponential and Related Functions in R}}
\subsection{Complex Exponential, Logarithms and Powers}

\clearpage
\section{Differentiation in One Variable}
\subsection{Differentiability}
\subsection{Rules for Differentiation}
\subsection{Mean Value Theorem}
\subsection{The Inequalities of Young, Hölder and Minkowski}
\subsection{L'Hospital's Rule}
\subsection{Taylor's Theorem}
\subsection{Iterative Procedures}
\subsection{Differentiation of Vector-valued Functions}

\clearpage
\section{Integration in One Variable}
\subsection{Piecewise Continuous Function}
\subsection{Banach Space of Piecewise Continuous Functions}
\subsection{Continuous Extensions}
\subsection{The Cauchy-Riemann Integral}
\subsection{The Riemann Sum}
\subsection{Properties of Integral}
\subsection{Integration Techniques}
\subsection{Sums}
\subsection{The two Fundamental Theorem of Calculus}
\subsection{Integration of Vector-valued Functions}
\subsection{Rectifiable Curves}

\clearpage
\section{Sequences and Series of Functions}
\subsection{Uniform Convergence}
\subsection{The Weierstrass Majorant Criterion}
\subsection{Continuity and Differentiability}
\subsection{Analytic Functions}
\subsection{\texorpdfstring{The Riemann $\zeta$-Function}{The Riemann zeta-Function}}
\subsection{Banach Algebras}
\subsection{The Stone-Weierstrass Theorem}
\subsection{Polynomial and Trigonometric Approximations}

\clearpage
\section{Fourier Analysis}
\subsection{\texorpdfstring{The $L_2$-Scalar Product}{The L2-Scalar Product}}
\subsection{Orthonormal Systems}
\subsection{Periodic Functions}
\subsection{Fourier Coefficients}
\subsection{Fourier Series}
\subsection{The Bessel Inequality}
\subsection{Complete Orthonormality}
\subsection{\texorpdfstring{Dirac $\delta$-Function}{Dirac delta-Function}}

\clearpage
\section{Improper Interals}
\subsection{Absolute Convergence Integral}
\subsection{Gamma Function}
\subsection{\texorpdfstring{Euler $\beta$-Integral}{Euler beta-Integral}}

\end{document}