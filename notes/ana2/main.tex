\documentclass[12pt]{article} % use larger type; default would be 10pt

\usepackage[utf8]{inputenc}

\usepackage{amsmath, amssymb, amsthm, mathtools}
\usepackage{microtype}

%%% Examples of Article customizations
% These packages are optional, depending whether you want the features they provide.
% See the LaTeX Companion or other references for full information.

%%% PAGE DIMENSIONS
\usepackage{geometry} % to change the page dimensions
\geometry{a4paper} % or letterpaper (US) or a5paper or....
\geometry{margin=2.5cm} % for example, change the margins to 2 inches all round
%\geometry{landscpe} % set up the page for landscape
%   read geometry.pdf for detailed page layout information

\usepackage{graphicx} % support the \includegraphics command and options

%\usepackage[parfill]{parskip} % Activate to begin paragraphs with an empty line rather than an indent

%%% PACKAGES
\usepackage{booktabs} % for much better looking tables
\usepackage{array} % for better arrays (eg matrices) in maths
\usepackage{paralist} % very flexible & customisable lists (eg. enumerate/itemize, etc.)
\usepackage{verbatim} % adds environment for commenting out blocks of text & for better verbatim
\usepackage{subfig} % make it possible to include more than one captioned figure/table in a single float
\usepackage[hidelinks]{hyperref}
\usepackage{csquotes}
% These packages are all incorporated in the memoir class to one degree or another...

%%% HEADERS & FOOTERS
\usepackage{fancyhdr} % This should be set AFTER setting up the page geometry
\pagestyle{fancy} % options: empty , plain , fancy
\renewcommand{\headrulewidth}{0pt} % customise the layout...
\lhead{}\chead{}\rhead{}
\lfoot{}\cfoot{\thepage}\rfoot{}

%%% SECTION TITLE APPEARANCE
\usepackage{sectsty}
%\allsectionsfont{\sffamily\mdseries\upshape} % (See the fntguide.pdf for font help)
% (This matches ConTeXt defaults)

%%% ToC (table of contents) APPEARANCE
%\usepackage[nottoc,notlof,notlot]{tocbibind} % Put the bibliography in the ToC
%\usepackage[titles,subfigure]{tocloft} % Alter the style of the Table of Contents
%\renewcommand{\cftsecfont}{\rmfamily\mdseries\upshape}
%\renewcommand{\cftsecpagefont}{\rmfamily\mdseries\upshape} % No bold!

% Theorem-like environments setup
\theoremstyle{plain}
\newtheorem{axiom}{Axiom}[section]
\newtheorem{theorem}{Theorem}[section]
\newtheorem{lemma}[theorem]{Lemma}
\newtheorem{proposition}[theorem]{Proposition}
\newtheorem{corollary}[theorem]{Corollary}

\theoremstyle{definition}
\newtheorem{definition}[theorem]{Definition}
\newtheorem{example}[theorem]{Example}
\newtheorem{remark}[theorem]{Remark}
\newtheorem{notation}[theorem]{Notation}

% Custom theorem style for important results
\newtheoremstyle{important}
  {\topsep}   % Space above
  {\topsep}   % Space below
  {\itshape}  % Body font
  {}          % Indent amount
  {\bfseries} % Theorem head font
  {.}         % Punctuation after theorem head
  {.5em}      % Space after theorem head
  {\thmname{#1}\thmnumber{ #2}\thmnote{ (#3)}} % Theorem head spec

\theoremstyle{important}
\newtheorem*{maintheorem}{Main Theorem}

% Remove extra indentation after environments
\AtEndEnvironment{axiom}{\noindent}
\AtEndEnvironment{theorem}{\noindent}
\AtEndEnvironment{lemma}{\noindent}
\AtEndEnvironment{proposition}{\noindent}
\AtEndEnvironment{corollary}{\noindent}
\AtEndEnvironment{definition}{\noindent}
\AtEndEnvironment{example}{\noindent}
\AtEndEnvironment{remark}{\noindent}
\AtEndEnvironment{notation}{\noindent}
\AtEndEnvironment{maintheorem}{\noindent}

%%% END Article customizations

\title{\textbf{Analysis II}}
\author{Harsh Prajapati}
\date{25.07.26}

\begin{document}
\maketitle

These notes were prepared between December 2025 and (tentative) \textbf{(Last update: \today)}

If you find any mistakes or typos, please report them to \textbf{caccacpenguin@gmail.com}. I would really appreciate it.

I often use informal language to make the ideas easier to grasp, but it's important to keep in mind the formalism and not get too attached to the informal ideas. My goal is to make the material feel approachable, while still respecting the rigour that makes mathematics what it is.

I hope you find these notes helpful :D!

\section*{Textbook Recommendations}

These books will serve as our main references:

\begin{itemize}
  \item Herbert Amann, Joachim Escher, Analysis II, Zweite Auflage, Birkhäuser-Verlag, 2006, Basel
  \item Otto Forster, Florian Lindemann, Analysis 2, 12. Auflage, 2025
  \item Walter Rudin, Principles of Mathematical Analysis, 3rd. Edition
\end{itemize}
 
Some other great resources.
 
\begin{itemize}
  \item K. Königsberger, Analysis 2, 2002
  \item W. Walter, Analysis 2, 2002
  \item Heuser, Lehrbuch der Analysis (Teil 2), 2002
  \item James R. Munkres, Topology, 2nd. Ed., 2000
\end{itemize}
\clearpage

\tableofcontents
\clearpage

\section{Introduction}
\clearpage

\section{Differential Calculus of Several Variables}

\subsection{Vector Spaces}

\subsection{Basis and Dimensions}

\subsection{Direct Sums}
\clearpage

\section{Curve Integral}

\subsection{Matrix Multiplication}

\subsection{Systems of Linear Equations}

\subsection{Matrices and Elementary Row Operations}

\subsection{Row-Reduced Echelon Matrices}

\subsection{Invertible Matrices}
\clearpage

\end{document}