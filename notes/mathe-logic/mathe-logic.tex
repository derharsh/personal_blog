\documentclass[12pt]{article} % use larger type; default would be 10pt

\usepackage[utf8]{inputenc}

\usepackage{amsmath, amssymb, amsthm, mathtools}
\usepackage{microtype}

%%% Examples of Article customizations
% These packages are optional, depending whether you want the features they provide.
% See the LaTeX Companion or other references for full information.

%%% PAGE DIMENSIONS
\usepackage{geometry} % to change the page dimensions
\geometry{a4paper} % or a5paper or....
\geometry{margin=2.5cm} % for example, change the margins to 2 inches all round
%\geometry{landscpe} % set up the page for landscape
%   read geometry.pdf for detailed page layout information

\usepackage{graphicx} % support the \includegraphics command and options

%\usepackage[parfill]{parskip} % Activate to begin paragraphs with an empty line rather than an indent

%%% PACKAGES
\usepackage{booktabs} % for much better looking tables
\usepackage{array} % for better arrays (eg matrices) in maths
\usepackage{paralist} % very flexible & customisable lists (eg. enumerate/itemize, etc.)
\usepackage{verbatim} % adds environment for commenting out blocks of text & for better verbatim
\usepackage{subfig} % make it possible to include more than one captioned figure/table in a single float
\usepackage[hidelinks]{hyperref}
\usepackage{csquotes, lmodern, microtype, makeidx, booktabs, array, paralist}
\usepackage{tikz-cd}
% These packages are all incorporated in the memoir class to one degree or another...

%%% HEADERS & FOOTERS
\usepackage{fancyhdr} % This should be set AFTER setting up the page geometry
\pagestyle{fancy} % options: empty , plain , fancy
\renewcommand{\headrulewidth}{0pt} % customise the layout...
\lhead{}\chead{}\rhead{}
\lfoot{}\cfoot{\thepage}\rfoot{}

%%% SECTION TITLE APPEARANCE
\usepackage{sectsty}
%\allsectionsfont{\sffamily\mdseries\upshape} % (See the fntguide.pdf for font help)
% (This matches ConTeXt defaults)

%%% ToC (table of contents) APPEARANCE
%\usepackage[nottoc,notlof,notlot]{tocbibind} % Put the bibliography in the ToC
%\usepackage[titles,subfigure]{tocloft} % Alter the style of the Table of Contents
%\renewcommand{\cftsecfont}{\rmfamily\mdseries\upshape}
%\renewcommand{\cftsecpagefont}{\rmfamily\mdseries\upshape} % No bold!

% Theorem-like environments setup
\theoremstyle{plain}
\newtheorem{theorem}{Theorem}[section]
\newtheorem{axiom}[theorem]{Axiom}
\newtheorem{lemma}[theorem]{Lemma}
\newtheorem{proposition}[theorem]{Proposition}
\newtheorem{corollary}[theorem]{Corollary}

\theoremstyle{definition}
\newtheorem{definition}[theorem]{Definition}
\newtheorem{example}[theorem]{Example}
\newtheorem{remark}[theorem]{Remark}

\newenvironment{solution}
  {\begin{proof}[Solution]}
  {\end{proof}}

% Remove extra indentation after environments
\AtEndEnvironment{theorem}{\noindent}
\AtEndEnvironment{lemma}{\noindent}
\AtEndEnvironment{proposition}{\noindent}
\AtEndEnvironment{corollary}{\noindent}
\AtEndEnvironment{definition}{\noindent}
\AtEndEnvironment{example}{\noindent}
\AtEndEnvironment{remark}{\noindent}
\AtEndEnvironment{notation}{\noindent}
\AtEndEnvironment{maintheorem}{\noindent}

%%% END Article customizations

\title{\textbf{Mathematical Logic}}
\author{Harsh Prajapati}
\date{30.03.26}

\begin{document}
\maketitle

These notes were prepared between January 2026 and (tentative) \textbf{(Last update: \today)}.

If you find any mistakes or typos, please report them to \textbf{caccacpenguin@gmail.com}. I would really appreciate it.

I often use informal language to make the ideas easier to grasp. My goal is to make the material feel approachable, while still respecting the rigor that makes mathematics what it is.

I hope you will find these notes helpful :D!

\subsection*{Textbook Recommendations}

These notes are closely based on H. Schwichtenberg's lecture notes from WiSe 2025-26 on \textit{Mathematische Logik} given at Mathematisches Institut, Ludwig-Maximilians-Universit\"at, M\"unchen. These books can serve as our main resources.

\begin{itemize}
  \item Stanley S. Wainer, H. Schwichtenberg, Proofs and Computations
  \item A.S. Troelstra, H. Schwichtenberg, Basic Proof Theory, 2nd. Ed., 2000.
  \item Dirk van Dalen, Logic and Structure, 5th Edition, 2013.
  \item Heinz-Dieter Ebbinghaus, Jörg Flum, Wolfgang Thomas, Einführung in die mathematische Logik, 6. Auflage, 2018.
\end{itemize}

Other great resources and furthur readings:

\begin{itemize}
  \item Stephen Cole Kleene, Introduction to Metamathematics, 1971.
  \item Joseph R. Shoenfield, Mathematical Logic, 1967.
  \item Joseph Mileti, Modern Mathematical Logic, 2013.
  \item Haskell B. Curry, Foundations of Mathematical Logic, 1977.
  \item Herbert B. Enderton, A Mathematical Introduction to Logic, 2nd. Ed., 2001.
\end{itemize}
\clearpage

\tableofcontents
\clearpage

\section{Introduction}
This course is an introduction to mathematical logic and the foundations of mathematics. It covers both proof theory and model theory, with a focus on formal systems, their syntax and semantics, and the relationship between them.

The \textbf{contents} of this course is minimal logic and the integration of classical and intuitionistic logic; Gentzen's natural deduction calculus. Semantics and completeness in first-order predicate logic. Foundations of computability: Church's thesis and the undecidability of predicate logic. G\"odel's incompleteness theorem for extensions of elementary number theory. Construction and structure of the number systems.

We'll be using Minlog, developed by Schwichtenberg's logic team at LMU Munich to formalise proofs and not something like Lean or Coq (cry more). The reason for this is because it's a low-level language so you're really learn the foundations of proof assistants well(but prof approval is more imp obv). To put that in perspective, if Lean or Coq are Python then Minlog is C. yes, youre bouta get cooked!!

The \textbf{prerequisites} for this course is some background in basic logic and set theory which you can find in the lecture notes on \textit{Mathematical Foundations}.
\clearpage

\section{Proof Theory}

\section{Recursion Theory}

\section{G\"odel's Incompletness Theorem}
\end{document}